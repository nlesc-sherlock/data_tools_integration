\section{Orchestration}
\label{orchestration}

We first use Spark since data is stored in HDFS.
So spark is used to exploit data locality and fault-tolerance.
Currently TDI team is working on the integration of other team’s tools using spark\_docker into a concise
workflow. For efficient resource utilization, efficient scheduling while co-existing with other applications
running on the same cluster, TDI team is looking at Mesos and Kubernete. Furthermore, for flexible heterogeneous
data integration and the use of external tables as input data is other direction under consideration where
the data injection stands on the same principles as our internal Data Vaults project.

The map of such workflow into the required resources should be delegated to an outside
systems with a complete overview of the existent ecosystem.

We schedule different docker containers through spark.
How to connect them using Spark?

Then how other scheduling schemas can be integrated?

Different types of schedulers:
%https://www.digitalocean.com/community/tutorials/the-docker-ecosystem-scheduling-and-orchestration

%https://docs.docker.com/swarm/scheduler/filter/

Data locality could be questioned and then by using swarm filters we could schedule dockers to
specific locations, however, the scheduling algorithm would be of our responsability.

Keubernet, Spark and Zepplin
%http://blog.kubernetes.io/2016/03/using-Spark-and-Zeppelin-to-process-Big-Data-on-Kubernetes.html

Comparisson between Mesos and Keubernet, fantastic list of differences and why we will pick Keubernet.
%http://thenewstack.io/a-brief-comparison-of-mesos-and-kubernetes/
