\section{System in Action}
\label{system_in_action}


\subsection{Deep Learning}
The Convolutional Neural Networks (CNN) are a type of deep learning networks, Y. Bengio, "Learning deep
architechures for AI". CNNs are a family of multi-layer neural networks particularly designed for use
on two-dimensional data, such as images. For a quick introduction to CNNs, please refer to section 2.3
of the “Large-scale Computer Vision” overview. For a more elaborate introduction, please refer to the
online book “Neural Networks and Deep Learning” and to the vast resources on deeplearning.net.
Sherlock

Image classification in forensics is a very common request by the digital forensic investigators. 

\subsection{Analyzing document corpora}
While searching a device for incriminating evidence, text documents are commonly encountered. Analyzing the
content of such documents is a challenging task which would require investigators to read through large volumes
of text, without any information to guide them through the search. What is this document about? How does it
relate to other documents? Are there any trends amongst them? Do I need to read through this pile of documents?
Or could I spot a particular document which is more interesting than the rest?

We will tackle these problems using natural language processing tools. The aim of this group is to use NLP
tools to group documents with similar content and provide investigators with the tools to quickly gain insight
into large collections of text.

In the field of Natural Language Processing (NLP), topic modeling has become a popular technique for analyzing
large sets of documents. Topic models attempt to infer a set of topics (basically, bags of words) and then
identify the probability of a topic being contained in a document (e.g. Document1 is made up of 0.5 * Football
topic + 0.2 * General sports topic + 0.3 * Drugs and crime topic).

However there are still many open questions when using topic modeling: document pre-processing, optimum number
of topics to use, visualization and interpretation of the results, etc.

One focus are is investigating techniques for visualizing results from topic models in the most useful and
meaningful way. The ultimate aim is to be able to take a list set of documents (e.g. emails from an inbox)
and cluster them into different groups of similar emails (e.g. emails work, family, leisure). To achieve this
we will need to use existing natural language processing technique, investigate suitable similarity metrics
between documents, and develop (interactive?) visualization techniques.

We will be using the infamous Enron email data set as our document collection.
