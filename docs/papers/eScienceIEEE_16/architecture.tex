\section{Architecture}
\label{architecture}

Types of scheduling.

We first use Spark since data is stored in HDFS.
So spark is used to exploit data locality and fault-tolerance.


We schedule different docker containers through spark.
How to connect them using Spark?

Then how other scheduling schemas can be integrated?

Different types of schedulers:
https://www.digitalocean.com/community/tutorials/the-docker-ecosystem-scheduling-and-orchestration

https://docs.docker.com/swarm/scheduler/filter/

Data locality could be questioned and then by using swarm filters we could schedule dockers to
specific locations, however, the scheduling algorithm would be of our responsability.

Keubernet, Spark and Zepplin

http://blog.kubernetes.io/2016/03/using-Spark-and-Zeppelin-to-process-Big-Data-on-Kubernetes.html


How to do it for HDFS cluster, we run docker containers in Spark.

Outside HDFS cluster we use keubernet to schedule docker containers over each node.
A spark cluster can be created, but also a set of containers running othere tools.
Using messos for resource management, several processing tools can co-exist at the same time.

The challenge becomes to access heterogenous data and find a common ground to share data.
The idea is to exploit data vaults when using a DBMS or Spark. Specialized libraries work
with a single data format. They are used often in data preparation phase where classificaiton happens.

Data vaults is not only about importing data, it is also about exporting data.

We are inverting the paradigm with this platform, it is not a DBMS on top of Spark or Hadoop, but both co-exist side by side
where the query plan abstraction goes on level up. With this query interaction is more user friendly, instead of gathering
information we are now extracting knowledge.

When using HDFS, better to use Spark because data locality comes for free.

Using Kubernet it is the first step towards running Spark with access to external file repositories.
For that we will use and extend the latest API released:

In Spark 1.2, we’ve taken the next step to allow Spark to integrate natively with a far larger number of input sources.  These new integrations are made possible through the inclusion of the new Spark SQL Data Sources API.
https://databricks.com/blog/2015/01/09/spark-sql-data-sources-api-unified-data-access-for-the-spark-platform.html

Comparisson between Mesos and Keubernet, fantastic list of differences and why we will pick Keubernet.
http://thenewstack.io/a-brief-comparison-of-mesos-and-kubernetes/


Tricks on how to run spark within containers and be scheduled by Keubernet:
http://www.lshift.net/blog/2015/04/30/swarming-spark/
