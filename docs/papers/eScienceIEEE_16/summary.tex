\section{Future work}

----Towards Generating ETL Processes for Incremental Loading---
We presented an approach for generating
incremental load
processes
for  data  warehouse  refreshment  from
declarative
schema mappings
.  The basis of our work has been provided
by Orchid, a prototype system developed at IBM Almaden
Research Center, that translates
schema mappings
into ETL
processes and vice versa.  Orchid-generated ETL processes,
however,  are  limited  to
initial  load  scenarios
,  i.e.  source
data  is  exhaustively  extracted  and  the  warehouse  is  com-
109
pletely  (re)built.   Incremental  load  processes,  in  contrast,
propagate changes from the sources to the data warehouse.
This approach has clear performance benefits.  Change Data
Capture (CDC) and Change Data Application (CDA) tech-
niques  are  used  to  capture  changes  at  the  sources  and  re-
fresh the data warehouse, respectively.  We defined a model
for
change  data
to  characterize  both,  the  output  of  CDC
techniques  and  the  input  of  CDA  techniques.   Since  CDC
techniques may su from limitations we introduced a no-
tion of
partial  change  data
.  We discussed the propagation
of  partial  change  data  within  incremental  load  processes.
Our  approach  allows  for  reasoning  on  how  limitations  of
CDC techniques determine the set of applicable CDA tech-
niques.  That is, it allows inferring satisticable CDA reuire-
ments from given CDC limitations and, the other way round,
acceptable CDC limitations from given CDA requirements.
We further, demonstrated the exploitation of properties of
data sources (such as schema constraints) to reduce the com-
plexity  of  incremental  load  processes.   By  leveraging  Or-
chid's  deployment  facilities,  we  are  able  to  generate  exe-
cutable  ETL  processes.   We  are  confident  that  our  work
contributes to the improvement of ETL development 




\section{Summary}
\label{summary}
